\documentclass{article}[12pt]
\usepackage{color}

\begin{document}


\hrule
\begin{center}\textbf{\Large{GlusterFS Coding Standards}}\end{center}
\begin{center}\textbf{\large{\textcolor{red}{Z} Research}}\end{center}
\begin{center}{July 14, 2008}\end{center}
\hrule

\vspace{8ex}

\section*{$\bullet$ Structure definitions should have a comment per member}

Every member in a structure definition must have a comment about its
purpose. The comment should be descriptive without being overly verbose.

\vspace{2ex}
\textsl{Bad}:

\begin{verbatim}
        gf_lock_t   lock;           /* lock */
\end{verbatim}

\textsl{Good}:

\begin{verbatim}
        DBTYPE      access_mode;    /* access mode for accessing 
                                     * the databases, can be
                                     * DB_HASH, DB_BTREE
                                     * (option access-mode <mode>)
                                     */
\end{verbatim}

\section*{$\bullet$ Declare all variables at the beginning of the function}
All local variables in a function must be declared immediately after the
opening brace. This makes it easy to keep track of memory that needs to be freed
during exit. It also helps debugging, since gdb cannot handle variables
declared inside loops or other such blocks.

\section*{$\bullet$ Always initialize local variables}
Every local variable should be initialized to a sensible default value
at the point of its declaration. All pointers should be initialized to NULL,
and all integers should be zero or (if it makes sense) an error value.

\vspace{2ex}

\textsl{Good}:

\begin{verbatim}
        int ret       = 0;
        char *databuf = NULL;
        int _fd       = -1;
\end{verbatim}

\section*{$\bullet$ Initialization should always be done with a constant value}
Never use a non-constant expression as the initialization value for a variable.

\vspace{2ex}

\textsl{Bad}:

\begin{verbatim}
        pid_t pid     = frame->root->pid;
        char *databuf = malloc (1024);
\end{verbatim}

\section*{$\bullet$ Validate all arguments to a function}
All pointer arguments to a function must be checked for \texttt{NULL}.
A macro named \texttt{VALIDATE} (in \texttt{common-utils.h}) 
takes one argument, and if it is \texttt{NULL}, writes a log message and
jumps to a label called \texttt{err} after setting op\_ret and op\_errno
appropriately. It is recommended to use this template.

\vspace{2ex}

\textsl{Good}:

\begin{verbatim}
        VALIDATE(frame);
        VALIDATE(this);
        VALIDATE(inode);
\end{verbatim}

\section*{$\bullet$ Never rely on precedence of operators}
Never write code that relies on the precedence of operators to execute correctly.
Such code can be hard to read and someone else might not know the precedence
of operators as accurately as you do.
\vspace{2ex}

\textsl{Bad}:

\begin{verbatim}
        if (op_ret == -1 && errno != ENOENT)
\end{verbatim}

\textsl{Good}:

\begin{verbatim}
        if ((op_ret == -1) && (errno != ENOENT))
\end{verbatim}

\section*{$\bullet$ Use exactly matching types}
Use a variable of the exact type declared in the manual to hold the return value
of a function. Do not use an ``equivalent'' type.

\vspace{2ex}

\textsl{Bad}:

\begin{verbatim}
        int len = strlen (path);
\end{verbatim}

\textsl{Good}:

\begin{verbatim}
        size_t len = strlen (path);
\end{verbatim}

\section*{$\bullet$ Never write code such as \texttt{foo->bar->baz}; check every pointer}
Do not write code that blindly follows a chain of pointer references. Any pointer in
the chain may be \texttt{NULL} and thus cause a crash. Verify that each pointer
is non-null before following it.

\section*{$\bullet$ Check return value of all functions and system calls}
The return value of all system calls and API functions must be checked for success
or failure.

\vspace{2ex}
\textsl{Bad}:

\begin{verbatim}
        close (fd);
\end{verbatim}

\textsl{Good}:

\begin{verbatim}
        op_ret = close (_fd);
        if (op_ret == -1) {
          gf_log (this->name, GF_LOG_ERROR, 
                  "close on file %s failed (%s)", real_path, 
                  strerror (errno));
          op_errno = errno;
          goto err;
        }
\end{verbatim}


\section*{$\bullet$ Gracefully handle failure of malloc}
GlusterFS should never crash or exit due to lack of memory. If a memory allocation
fails, the call should be unwound and an error returned to the user.

\section*{$\bullet$ Use result args and reserve the return value to indicate success or failure}
The return value of every functions must indicate success or failure (unless 
it is impossible for the function to fail --- e.g., boolean functions). If 
the function needs to return additional data, it must be returned using a 
result (pointer) argument.

\vspace{2ex}
\textsl{Bad}:

\begin{verbatim}
        int32_t dict_get_int32 (dict_t *this, char *key);
\end{verbatim}

\textsl{Good}:

\begin{verbatim}
        int dict_get_int32 (dict_t *this, char *key, int32_t *val);
\end{verbatim}

\section*{$\bullet$ Always use the `n' versions of string functions}
Unless impossible, use the length-limited versions of the string functions.

\vspace{2ex}
\textsl{Bad}:

\begin{verbatim}
        strcpy (entry_path, real_path);
\end{verbatim}

\textsl{Good}:

\begin{verbatim}
        strncpy (entry_path, real_path, entry_path_len);
\end{verbatim}

\section*{$\bullet$ No dead or commented code}
There must be no dead code (code to which control can never be passed) or 
commented out code in the codebase.

\section*{$\bullet$ Only one unwind and return per function}
There must be only one exit out of a function. \texttt{UNWIND} and return 
should happen at only point in the function.

\section*{$\bullet$ Keep functions small}
Try to keep functions small. Two to three screenfulls (80 lines per screen) is
considered a reasonable limit. If a function is very long, try splitting it
into many little helper functions.

\vspace{2ex}
\textsl{Example for a helper function}:
\begin{verbatim}
        static int
        same_owner (posix_lock_t *l1, posix_lock_t *l2)
        {
          return ((l1->client_pid == l2->client_pid) &&
                  (l1->transport  == l2->transport));
        }
\end{verbatim}

\section*{A skeleton fop function}
This is the recommended template for any fop. In the beginning come the
initializations. After that, the `success' control flow should be linear. 
Any error conditions should cause a \texttt{goto} to a single point, \texttt{err}.
At that point, the code should detect the error that has occured and do
appropriate cleanup.

\begin{verbatim}
int32_t 
sample_fop (call_frame_t *frame,
            xlator_t *this,
            ...)
{
  char *var1 = NULL;
  int32_t op_ret = 0;
  int32_t op_errno = 0;
  DIR *dir = NULL;
  struct posix_fd *pfd = NULL;

  VALIDATE(frame);
  VALIDATE(this);

  /* other validations */
  
  dir = opendir (...);

  if (dir == NULL) {
    op_ret = -1;
    op_errno = errno;
    gf_log (this->name, GF_LOG_ERROR, 
            "opendir failed on %s (%s)", loc->path, strerror (op_errno));
    goto err;
  }

  /* another system call */
  if (...) {
    op_ret = -1;
    op_errno = ENOMEM;
    gf_log (this->name, GF_LOG_ERROR,
            "out of memory :(");
    goto err;
  }

  /* ... */

 err:
  if (op_ret == -1) {

    /* check for all the cleanup that needs to be
       done */

    if (dir) {
      closedir (dir);
      dir = NULL;
    }
    if (pfd) {
      if (pfd->path)
        FREE (pfd->path);
      FREE (pfd);
      pfd = NULL;
    }
  }

  STACK_UNWIND (frame, op_ret, op_errno, fd);
  return 0;
}
\end{verbatim}

\end{document}
